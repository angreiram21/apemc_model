\documentclass[12pt]{report}
%\documentclass[aps,preprint,showpacs,superscriptaddress,
%groupedaddress,amsmath,amssymb]{revtex4}
\usepackage{graphicx}
\usepackage{dcolumn}
\usepackage{bm}
\usepackage{color}
\usepackage[colorlinks=true, allcolors=blue]{hyperref}
%\usepackage{subfig}
\usepackage{csquotes}
%\usepackage{diagbox}
%\usepackage{lipsum}
%\usepackage{setspace}
\newcommand{\be}{\begin{equation}}
\newcommand{\ee}{\end{equation}}
\newcommand{\ba}{\begin{eqnarray}}
\newcommand{\ea}{\end{eqnarray}}
\newcommand{\bd}{\begin{displaymath}}
\newcommand{\ed}{\end{displaymath}}
\newcommand{\bea}{\begin{eqnarray}}
\newcommand{\eea}{\end{eqnarray}}
\newcommand{\di}{{\rm d}}
\renewcommand{\vec}[1]{\mbox{\boldmath$#1$}}
%\DeclareMathOperator{\Artanh}{Artanh}
%\DeclareMathOperator{\Arsinh}{Arsinh}

\begin{document}

\title{A toy model for simulating $\alpha$ particle emissions in p + 11B reactionns at $K_p=$10 MeV via Monte Carlo method}

\author{Angel Reina Ramirez, V. Magas}
\smallskip

%\affiliation{
%$^1$Departament de Fisica Quantica i Astrofisica,  
%Universitat\! de\! Barcelona,  Martí i Franquès 1, 08028 Barcelona, Spain\\
%$^2$Institut de Ciències del Cosmos,  
%Universitat\! de\! Barcelona,  Martí i Franquès 1, 08028 Barcelona, Spain\\
%$^3$Institute of Physics and Technology, University of Bergen,
%Allegaten 55, 5007 Bergen, Norway\\
%$^4$Frankfurt Institute for Advanced Studies (FIAS), Ruth-Moufang-Str. 1, 60438, %Frankfurt am Main, Germany\\
%$^5$Wigner Research Centre for Physics (RCP), XII. Konkoly Thege Miklós út 29-33, %Postbox 49, 1121 Budapest, Hungary\\
%$^6$Los Alamos National Laboratory, Los Alamos, 87545 New Mexico, USA
%}
\maketitle
\tableofcontents
%\acknowledgements

\begin{abstract}
We present a toy model for simulating $\alpha$ particle emissions in p$+$11B reactions at $K_p =$ 10 MeV via Monte Carlo method.  
Incident proton is accelerated along the z-axes by an oscillating electric field of frequency $\omega$, and its momentum is randomly generated at the moment of the collision. After collision we get an $\alpha$ particle, emitted in z-direction, and an excited 8Be nucleus at rest, which breaks down in two $\alpha$ particles emitted in opposite direction. Pseudorandom number generator Mersenne twister engine algorithm is employed to randomly spawn the emisson angles of $\alpha$ particles. 
\end{abstract}
%\date{\today}

%\pacs{25.75.-q, 24.70.+s, 47.32.Ef}

\chapter{Model description}
Incident proton is accelerated along the z-axes by an oscillating electric field of a given frequency $\omega$:
\begin{equation}
\vec{E} = (0,0,E_0cos(\omega t + \phi)).
\end{equation}

From Newton's third law we get proton momentum:
\begin{equation}
F_z = e E_0cos(\omega t + \phi) = \frac{dp_z}{dt},
\end{equation}
\begin{equation}
p_z = \frac{eE_0}{\omega} sin(\omega t + \phi) = p_{z,max} sin(\omega t + \phi),
\end{equation}
where $p_{z,max} = \sqrt{2M_pK_p}$.

$\omega t + \phi$ angle is randomly generated  via Mersenne twister engine algorithm at the time of the collision.


\section{INITIAL STATE}

\subsection{GESRM}



\appendix{Appendix I}

\end{document}
